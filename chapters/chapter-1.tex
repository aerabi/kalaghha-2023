%%%%%%%%%%%%%Chapter-1%%%%%%%%%%%%%%%%%


\chapter{فضاهای اندازه}



\section{اندازه‌ی لبگ یک بعدی}
 
 فرض کنیم که 
$(a, b)$ 
 یک بازه‌ی باز در 
 $\R$ 
 باشد. حال اندازه‌ی لبگ آن را قرار می‌دهیم:
\begin{align*}
I=(a, b) \hspace{2cm} \lambda(I) := b - a
 \end{align*}
 
 و برای هر مجموعه‌ی باز دلخواه دیگر مانند 
 $ G \subseteq \R$
 قرار می‌دهیم:
 \begin{equation*}
 \lambda(G) = \sum_{k \in \N} \lambda(I_{k}) 
 \end{equation*}
 
 که در آن‌ها 
 $ \lbrace I_k \rbrace $
 ها بازه‌هایی هستند که مجموعه‌ی باز 
 $ G $
 را به وجود می‌آورند. اگر یکی از آنها نامتناهی باشد، آنگاه می‌گوییم که 
 $ \lambda(G) = $
 اندازه‌ی لبگ 
 $ G $
 بی‌نهایت است. و اگر 
 $ G \neq \phi $
، آنگاه 
$ \lambda(G) = 0 $
در ادامه اگر 
$ E $
مجموعه‌ای بسته و کراندار باشد و 
$ [a, b] $
کوچکترین بازه‌ی ؟ 
$ E $
آنوقت اندازه ی لبگ 
$ \lambda $
در ؟ زیر صدق می‌کند.
\begin{equation*}
\lambda(E) = b - a - \lambda ((a, b) \ E ) 
\end{equation*}
و یا
\begin{equation*}
\lambda(E) + \lambda ((a, b) \ E) = b - a 
\end{equation*}

\begin{definition}[\bf اندازه‌ی خارجی لبگ یک بعدی]
فرض کنیم که 
$ E $
مجموعه‌ای دلخواه از 
$ \R $
باشد. حال اندازه‌ی خارجی آن را با نماد 
$ {\lambda}*(E) $
نشان می‌دهیم و عبارت است از 
\begin{equation*}
{\lambda}*(E) = \inf \lbrace \lambda(G) \vert E \subset G , G \rbrace 
\end{equation*}
و یا با توجه به تعریف:
\begin{align}
= \inf \lbrace \sum_{k = 1}^{\infty} \lambda(I_k) : E \subset \bigcup_{k = 1}^{\infty} I_k , ? \rbrace \nonumber
\end{align}
\end{definition}

\begin{remark}[\bf تمرین]
تعریف بالا خوشتعریف بوده و اگر
$ E_1 \subset E_2 $
آنگاه 
$ {\lambda}*(E_1) \leq {\lambda}*(E_2) $
خاصیت 
$ monotonicity $ 
صدق می‌کند.

\end{remark}
\begin{remark}
اگر 
$E$
یک بازه‌ی باز یا شبه باشد، می‌توان نشان داد که 
$ {\lambda}*(E) = \lambda(E) $
است.
\end{remark}
\begin{remark}
$ {\lambda}* $
تعریف شده در بالا اگر 
$ E_1 $
و 
$ E_2 $
دو مجموعه‌ی مجزا باشد، آنگاه نمی‌توان گفت که 
\begin{align}
{\lambda}*(E_{1} \cup E_{2}) = {E_1}* + {E_2}* \nonumber
\end{align}
(ممکن است یکی از 
$ E_i $
ها اندازه‌پذیر لبگ نباشد.(
\end{remark}
\begin{definition}[\bf اندازه‌ی درونی لبگ یک بعدی]
فرض کنیم که 
$ E \subseteq \mathbb{R} $
باشد. حال قرار می‌دهیم
\begin{align}
{\lambda}_*(E) = \sup \lbrace \lambda(F) \mid F \subset E ,  F \rbrace \nonumber
\end{align}
(
$ \lambda(F) $
تعریف می‌شود زیرا فشرده است و با بازها پوشیده شده است.)
\end{definition}
\begin{remark}
اندازه‌ی درونی تعریف شده در صفحه ی قبل در خاصیت 
$ monotone $
صدق می‌کند یعنی اگر 
\begin{equation*}
E_1 \subset E_2 \ \ \Rightarrow \ \ {\lambda}_*(E_1) \leq {\lambda}_*(E_2)
\end{equation*}
\end{remark}
\begin{example}
فرض کنید که 
$ E = Q \cap [0, 1] $
ثابت کنید که 
$ {\lambda}^*(E) = 0 $
میشود.
\end{example}
برای این منظور فرض کنیم که
$ 0 = q_{0}, q_{1}, q_{2}, ... $
همه ی اعداد گویا در بازه‌ی 
$ [0, 1] $
باشند، به وضوح شمارش پذیر می‌باشند. لذا آنها را در دنباله‌ی 
$ {\lbrace q_n \rbrace}_{n = 1}^{\infty} $
قرار می‌دهیم. حال 
\begin{align}
\forall n \in \mathbb{N} \ \ \ I_n := (q_n - \frac{\epsilon}{2^n}, q_n + \frac{\epsilon}{2^n}) \ \ \ \lambda(I_n) = \frac{2 \epsilon}{2^n} = \frac{\epsilon}{2^{n - 1}} \nonumber
\end{align}
\begin{equation*}
\Rightarrow \forall \epsilon \> 0  \ \ \ \exists {\lbrace I_n \rbrace}_{n = 1}^{\infty}  s.t  E \subset \bigcup_{n = 1}^{\infty} I_n 
\end{equation*}
\begin{align}
{\lambda}*(E) \leq \lambda(\bigcup_{n = 1}^{\infty} I_n) = \sum_{n = 1}^{\infty} \lambda(I_n) = \epsilon \sum \frac{1}{2^{n - 1}} < \epsilon \nonumber
\end{align}
\begin{equation*}
 {\lambda}*(E) < \epsilon \Rightarrow {\lambda}*(E) = 0
\end{equation*}
\begin{definition}[\bf جبر]
فرض کنیم که 
$ X $
یک مجموعه‌ی دلخواه باشد. حال یک خانواده از زیر مجموعه‌های 
$ X $
مانند 
$ \mathcal{A} $
را یک جبر برای 
$ X $
می‌گویند هرگاه در شرایط زیر صدق کند.
\begin{itemize}
\item [الف]$X \in \mathcal{A} \ \ \ (\varphi \in \mathcal{A}) $
\item [ب] $A \in \mathcal{A} \Rightarrow {A}^c = X \backslash A \in \mathcal{A}$ 
\item [ج] $if A, B \in \mathcal{A} \Rightarrow A \cup B \in \mathcal{A} $
\end{itemize}
و نتیجه می‌دهد 
\begin{equation*}
\bigcup^{n}_{i = 1} A_i \in \mathcal{A} \ \ \ if A_i \in \mathcal{A}
\end{equation*}
\end{definition}
\begin{definition}[\bf $ \sigma $- جبر]
خانواده‌ی 
$ \mathcal{A} $ 
از زیر‌ مجموعه‌های 
$ X $
را یک 
$ \sigma $-
جبر برای
$ X $
هرگاه 
$ \mathcal{A} $
یک جبر بوده و در شرط زیر صدق کند
\begin{equation*}
\forall \ \ \ \lbrace A_i \in \mathcal{A} : i \in \mathbb{N} \rbrace \ \ \bigcup^{\infty}_{i = 1} A_i \in \mathcal{A}
\end{equation*}
نسبت به اجتماع شمارش‌پذیر اعضا بسته باشد.
\end{definition}
\begin{remark}
با توجه به این که در جبر و 
$ \sigma $-
جبر ؟ دو عنصر موجود است می‌توان گفت که در اشتراک تعداد متناهی اعضا نیز وجود دارد زیرا 
\begin{equation*}
if \ \ A_i \in \mathcal{A} \ \ 1 \leq i \leq n \ \Rightarrow A_i^c \in \mathcal{A} \Rightarrow \bigcup^{n}_{i = 1} A_i^c \in \mathcal{A}
\end{equation*}
و یا
\begin{equation*}
(\cap A_i)^c \in \mathcal{A} \Rightarrow {(\cap A_i)^c}^c \in \mathcal{A} \ \ or \ \ A_i \in \mathcal{A}
\end{equation*}
به‌ همین ترتیب می‌توان نشان داد که در 
$ \sigma $-
جبر
\begin{equation*}
if \ \ \ {\lbrace A_i \rbrace}^{\infty}_{i = 1} \in \mathcal{A} \Rightarrow \bigcap^{\infty}_{i = 1} A_i \in \mathcal{A}
\end{equation*}
\end{remark}
\begin{example}
کوچکترین جبر برای مجموعه‌ی 
$ X $ 
عبارت است از 
$ \lbrace \phi , x \rbrace $
و بزرگترین جبر و یا
$ \sigma $-
جبر برای آن عبارت است از 
$ 2^X $
و یا همان
$ \rho(X) $
\end{example}
\begin{example}
اگر 
$ X $
یک مجموعه‌ی دلخواه باشد ناشمارا و قرار دهیم 
\begin{equation*}
\mathcal{A} = \lbrace E \subset X : ? \rbrace
\end{equation*}
آنگاه 
$ \mathcal{A} $ 
در بالا یک 
$ \sigma $-
جبر می‌باشد. خواص زیر را چک می کنیم:
\begin{equation*}
\begin{split}
1 : & \ \ X^c = \phi \ \ \ \textit{شمارش پذیر} \Rightarrow X \in \mathcal{A} \\
2 : & \ \ if \ \ A \in \mathcal{A} \Rightarrow \textit{شمارش پذیر} A^c \textit{یا} A \ \ \Rightarrow \textit{شمارش پذیر} A \textit{و یا} A^c \ \ \Rightarrow A^c \in \mathcal{A} \\
3 : & \ \ if \ \ A_i \in \mathcal{A} \ \ \Rightarrow \textit{شمارش پذیر} A_i^c \textit{و یا} A_i 
\end{split}
\end{equation*}
حال چرا 
$ \bigcup^{\infty}_{i = 1} A_i $
شمارش پذیر است؟

$ \bigcup A_i $
ها اگر شمارش پذیر باشند آنگاه همه‌ی 
$ A_i $
ها شمارش پذیرند در غیر این صورت
\begin{equation*}
\begin{split}
\exists k \ \ \ s.t \ \ 1 \leq k \leq n \ \ \ \textit{شمارش ناپذیر} A_k^c  \longrightarrow \ \ \textit{چون} A_k \in \mathcal{A} \ \ \Rightarrow \textit{باید} A^{c}_{k} \in \mathcal{A} 
\end{split}
\end{equation*}
یعنی 
$ A_k^c $
شمارش پذیر است. حال 
\begin{equation*}
{(\bigcup^{\infty}_{i = 1} A_i)}^c =  \bigcap^{\infty}_{i = 1}A_i^c  \subseteq A_k^c
\end{equation*}
و 
$ A_k^c $
شمارش پذیر بود لذا باید 
$ (\bigcup^{\infty}_{i = 1} A_i)^c $
شمارش پذیر باشد و این می‌گوید که 
$ \mathcal{A} $
یک 
$ \sigma $-
جبر می‌باشد. 
\end{example}

\title{
$  \sigma $-
جبر تولید شده توسط یک زیر مجموعه از 
$ X $ :
}

به راحتی می‌توان دید که اگر 
$ \mathcal{A} $
و 
$ \mathcal{B} $
دو 
$ \sigma $-
جبر باشند آن‌وقت 
$ \mathcal{A} \cap \mathcal{B} $
نیز یک 
$ \sigma $- 
جبر بر روی مجموعه‌ی 
$ X $
خواهد بود. لذا به طور استاندارد اگر 
$ E \subset X $
یک زیر مجموعه‌ی دلخواه باشد، 
$\sigma $- 
جبر تولید شده توسط خانواده‌ی 
$ E $
عبارت است از اشتراک تمامی 
$ \sigma $-
جبرهای که شامل 
$ E $
هستند یعنی 
\begin{equation*}
E \textit{-جبر تولید شده توسط}\sigma = \bigcap_{\mathcal{A} \textit{یک} {\sigma}-\textit{جبر و} E \subseteq \mathcal{A} \textit{است}} \mathcal{A}
\end{equation*}
\begin{remark}
اگر 
$ \mathcal{A}_1 $
و 
$ \mathcal{A}_2 $
$ \sigma $-
جبر باشد لزومی ندارد اجتماع آنها 
$ \sigma $-
جبر شود.
\end{remark}
برای مثال اگر 
$ X $
دلخواه و 
$ A \subseteq X $ 
یک زیرمجموعه‌ی دلخواه باشد،
$ \sigma $-
جبر تولید شده توسط 
$ A $
عبارت است از 
\begin{equation*}
A \textit{ توسط تولیدی -جبر} \sigma \mathcal{A} = \lbrace X, {\phi}, A, A^c \rbrace
\end{equation*}
\begin{definition}[\bf اندازه]
فرض کنیم که 
$ \mathcal{A} $
یک
$ \sigma $-
جبر روی مجموعه‌ی 
$ X $
باشد. حال تابعی چون 
$ \mu : \mathcal{A} \rightarrow [0, \infty] $ 
را که دارای خواص زیر است، یک اندازه روی 
$ \mathcal{A} $
گویند.
\begin{equation*}
\begin{split}
1 : & \mu (\phi) = 0 \\
2: & \mu (\bigsqcup_{i \textit{شمارا}} A_i) = \sum_{i \textit{شمارا}} \mu (A_i) \ \ \ \ A_i \in \mathcal{A} 
\end{split}
\end{equation*}
\end{definition}
اعضای 
$ \mathcal{A} $
را نسبت به 
$ \mu $
اندازه‌پذیر گویند و یه سه تایی 
$ (X, \mathcal{A}, \mu) $
یک فضای اندازه می‌گوییم.
\begin{remark}
در بعضی از موارد به 
$ (X, \mathcal{A}) $
فضای اندازه‌پذیر و به 
$ (X, \mathcal{A}, \mu) $
فضای اندازه‌ی ؟ نیز گویند.
\end{remark}
\begin{remark}
اگر 
$ \mu(X) < \infty $
شود، اندازه‌ی 
$ \mu $
متناهی نامیده می‌شود و اگر 
$ \mu (X) = 1 $
باشد به آن یک فضای احتمال می‌گوییم در حالتی که 
\begin{equation*}
\exists E_i \in \mathcal{A} \ \ s.t \ \ \mu (E_i) < \infty , X = \bigcup^{\infty}_{i = 1} E_i
\end{equation*}
آن‌وقت گوییم که 
$ \mu $
یک فضای اندازه‌ی 
$ \sigma $-
متناهی می‌باشد. در حالت کلی‌تر مجموعه‌ی 
$ E $
را نسبت به 
$ \mu $،
$ \sigma $-
متناهی گویند. اگر بتوان 
$ E $
را به صورت اجتماع مجموعه‌های  با اندازه‌ی متناهی نوشت، توجه می‌کنیم که اگر به ازای هر
$ E \in \mathcal{A} $
که 
$ \mu (X) = \infty $، 
مجموعه‌ای چون 
$ F \in \mathcal{A} $
موجود باشد که 
$ F \subset E $
و 
$ 0 < \mu (F) < \infty $
آنگاه 
$ \mu $
را نیمه متناهی گویند و می‌توان نشان داد که هر اندازه‌ی 
$ \sigma $-
متناهی، نیمه متناهی می‌باشد ولی عکس آن برقرار نمی‌باشد. (مثال پیدا کنید)
\end{remark}
 نتایجی که از یک تابع اندازه می‌توان گرفت فرض کنیم که 
$ (X, \mathcal{M}, \mu) $ 
یک فضای اندازه باشند. آنگاه دارایی خواص زیر را به راحتی می‌توان چک کرد.
\begin{equation*}
\begin{split}
1 : & if \ \ A, B \in \mathcal{M}, \ \ A \subset B \Rightarrow \mu(A) \leq \mu(B) \\
2 : & \mu (A \cup B) + \mu (A \cap B) = \mu(A) + \mu(B) \ \ \ \forall A, B \in \mathcal{M} \\
3 : & \mu(X \backslash A) = \mu(X) - \mu(A) \\
4 : & \forall A_i \in \mathcal{M}  \Rightarrow \mu(\bigcup_{i = 1}^{\infty}) \leq \sum_{i = 1}^{\infty} \mu(A_i) 
\end{split}
\end{equation*}

\begin{remark}
\begin{equation*}
\begin{split}
\bigcup B_i = \bigcup A_i \ \ if  B_1 = A_1, \ \ B_2 = A_2 \backslash A_1, \ \ B_3 = A_3 \backslash A_2 = A_3 \backslash (A_1 \cup A_2), \ldots \\
\bigcup B_i \in \mathcal{M} \ \ \ \mu(\bigcup B_i) = \sum^{\infty}_{i = 1} \mu(B_i) \leq \sum^{\infty}_{i = 1} \mu(A_i) ( B_i \subseteq A_i \Rightarrow \mu(B_i) \leq \mu(A_i)) \\
\end{split}
\end{equation*}
\end{remark}

\begin{theorem}
فرض کنیم که 
$ (X, \mathcal{M}, \mu) $
یک فضای اندازه و 
$\{ A_i \}\subset \mathcal M$
دنباله‌ای از مجموعه های اندازه پذیر باشد، آنگاه:
\begin{itemize}
\item[{[{\bf 1}]}]
اگر $\{A_i\}$ یک دنباله صعودی نسبت به رابطه شمول باشد، یعنی
$A_1 \subset A_2 \subset A_3 \subset \ldots $
آنگاه 
\begin{align*}
\lim_{n \to \infty} \mu(A_n) &= \mu\Big(\lim_{n \to \infty} A_n\Big)\\
& = \mu\Big(\bigcup_{n = 1}^{\infty} A_n\Big)
\end{align*}
\item[{[{\bf 2}]}]
اگر $\{A_i\}$ یک دنباله نزولی نسبت به رابطه شمول باشد، یعنی
$A_1 \supset A_2 \supset A_3 \supset \ldots $
و $\mu(A_m)<\infty$ برای یک اندیس مانند $m\in\N$، آنگاه 
\begin{align*}
\lim_{n \to \infty} \mu(A_n) &= \mu\Big(\lim_{n \to \infty} A_n\Big)\\
& = \mu\Big(\bigcap_{n = 1}^{\infty} A_n\Big)
\end{align*}
\end{itemize}
\end{theorem}

\pf
قرار می‌دهیم 
$ A := \bigcup_{i = 1}^{\infty} A_i $
بدیهی است که 
\begin{equation*}
\begin{split}
A = A_1 \sqcup (A_2 \backslash A_1) \sqcup (A_3 \backslash (A_1 \cup A_2)) \sqcup \ldots \Rightarrow \\
A = (A_1 \backslash A_0) \sqcup (A_2 \backslash A_1) \sqcup (A_3 \backslash A_2) \sqcup \ldots \\
\end{split}
\end{equation*}
توجه کنیم که ؟ ها مجزا از یکدیگر می باشند. بنابر خاصیت صعودی مجموعه ها حال:
\begin{equation*}
\begin{split}
\mu(A) =& \mu(\bigcup_{i = 1}^{\infty} (A_i \backslash A_{i-1})) = \sum_{i = 1}^{\infty} \mu(A_i \backslash A_{i - 1}) \\
           =& \sum_{i = 1}^{\infty} \mu(A_i) - \mu(A_{i - 1}) = \mu(A_0) - \mu(A_1) + \ldots + \mu(A_{n - 1}) - \mu(A_n) \\
           =& \mu(\bigcup_{i = 1}^{\infty} A_i) = \lim_{n \rightarrow \infty} \mu(A_n) \\
\end{split}
\end{equation*}

(توجه : اگر
$ A \subseteq B $
باشد، آنوقت:
$ \mu(B \backslash A) = \mu(B) - \mu(A) $
)

\pf
رقسمت دوم: برای ؟ قرار می‌دهیم
$ B_n := A_m \backslash A_n \textit{یا} A_m \cap A_n^c $
چون 
$ A_1^c \subset A_2^c \subset \ldots $
به وضوح
$ B_1 \subset B_2 \subset B_3 \subset \ldots $
از طرفی برای هر 
$ n \leq m $:
\begin{equation*}
\mu(B_n) = \mu(A_m \backslash A_n) = \mu(A_m) - \mu(A_n)
\end{equation*} 
توجه می‌کنیم که 
$ \mu(\bigcap A_i) \leq \mu(A_n) \leq \mu(A_m) < \infty $:
\begin{equation*}
\begin{split}
\lim_{n \rightarrow \infty} \mu(B_n) = \mu(\bigcup_{n = 1}^{\infty} B_n) = \mu(\bigcup_{n = 1}^{\infty} (A_m \cap A_n^c)) \\
\lim_{n \rightarrow \infty} [\mu(A_m) - \mu(A_n)] = \mu(A_m \cap (\bigcup_{n = 1}^{\infty} A_n^c)) = \mu(A_m \backslash (\bigcap_{n = 1}^{\infty} A_n)) \\
\Rightarrow \mu(A_m) - \lim_{n \rightarrow \infty} \mu(A_n) = \mu(A_m) - \mu(\bigcap_{n = 1}^{\infty} A_n) \\
\Rightarrow \lim_{n \rightarrow \infty} \mu(A_n) = \mu(\bigcap_{n = 1}^{\infty} A_n)
\end{split}
\end{equation*}

\begin{example}[\bf اندازه]
فرض کنیم که 
$ (X, \mathcal{M}) $
یک 
$ \sigma $-
جیر باشد، حال 
$ \mu $
به صورت زیر یک اندازه است:
\begin{equation*}
\begin{split}
\forall A \in \mathcal{M} \ \ \ \mu(A) := ? \\
X \neq \phi , a_0 \in X \ \ \ \mu(A) = 
\end{split}
\end{equation*}
\end{example}
\begin{example}
برای 
$ f \geq 0 $
انتگرال‌پذیر و 
$ (X, \mathcal{M}, \mu) $
را به صورت زیر تعریف می‌کنیم:
\begin{equation*}
A \in \mathcal{M} \ \ \ \ \mu(A) := \int_{A} f dx
\end{equation*}
\end{example}

\begin{definition}
فرض کنیم که 
$ \lbrace A_n \rbrace $
دنبالهای از زیر مجموعه‌های 
$ X $
باشد. تعریف می‌کنیم:
\begin{equation*}
\begin{split}
\limsup A_n = \bigcap_{m = 1}^{\infty} (\bigcup_{n = m}^{\infty} A_n) = \overline{\lim} A_n \\
\liminf A_n = \bigcup_{m = 1}^{\infty} (\bigcap_{n= m}^{\infty} A_n) = \underline{\lim} A_n \\
\end{split}
\end{equation*}
اگر 
$ A = \limsup_{n \rightarrow \infty} A_n = \liminf_{n \rightarrow \infty} A_n $
باشد آنوقت گوییم که دنباله‌ی 
$ \lbrace A_n \rbrace $
همگرا به 
$ A $
می‌باشد و با علامت 
$ \lim_{n \rightarrow \infty} A_n = A $
نمایش داده می‌شود.
\end{definition}

\begin{remark}
ثابت کنید که اگر 
$ A_i \in \mathcal{M} $
باشد، آنوقت 
$ \limsup_{n \rightarrow \infty} A_n $
و
$ \liminf_{n \rightarrow \infty} A_n $
دوباره در 
$ \mathcal{M} $
می‌باشند.
\end{remark}

\begin{recall}
برای دنباله‌ی عددی 
$ \lbrace x_n \rbrace_{n = 1}^{\infty} $
نمادهای 
$ \limsup_{n \rightarrow \infty} x_n $
حد زبرین و
$ \liminf_{n \rightarrow \infty} x_n $
حد زیرین می‌باشند.
\begin{equation*}
\begin{split}
\limsup_{n \rightarrow \infty} x_n = \overline{\lim_{n \rightarrow \infty}} x_n = \inf_{n} \sup_{m \geq n} x_m \\
\liminf_{n \rightarrow \infty} x_n = \underline{\lim_{n \rightarrow \infty}} x_n = \sup_{n} \inf_{m \geq n} x_m \\
\end{split}
\end{equation*}
خواص آن:
\begin{equation*}
\begin{split}
1 :& \ \liminf_{n \rightarrow \infty} x_n \leq \limsup_{n \rightarrow \infty} x_n \\
2 :& \ \liminf_{n \rightarrow \infty} x_n =  - \limsup_{n \rightarrow \infty} (- x_n) \\
3 :& \ \liminf_{n \rightarrow \infty} x_n + \liminf_{n \rightarrow \infty} y_n \leq \liminf_{n \rightarrow \infty} (x_n + y_n) \\
    &\leq \liminf_{n \rightarrow \infty} x_n + \limsup_{n \rightarrow \infty} y_n \leq \limsup_{n \rightarrow \infty} x_n + \limsup_{n \rightarrow \infty} y_n  \\
\end{split}
\end{equation*}
\end{recall}

\begin{lemma}[برل - کانتلی]
فرض کنیم که 
$ (X, \mathcal{M}, \mu) $
فضای اندازه‌ای باشد که 
$ A_i \in \mathcal{M} $
و
$ \sum_{i = 1}^{\infty} \mu(A_i) < \infty $
آنگاه 
$ \mu(\limsup_{n \rightarrow \infty} A_n) = 0 $
\end{lemma}