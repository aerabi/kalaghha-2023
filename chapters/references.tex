 % مراجع خود را در این قسمت وارد کنید.
% دستوری برای کوچک کردن اندازه فونت‌ها 
%\small
% شروع محیط مراجع 
\setLTRbibitems
\begin{thebibliography}{99}
\resetlatinfont
% تذکر: چنانچه از نسخه‌های قدیمی زی‌پرشین و بی‌دی استفاده می‌کنید، هنگام استفاده از سه دستور بالا، با خطا 
% مواجه می‌شوید. بنابراین باید زی‌پرشین و بی‌دی خود را آپدیت کنید.
% چنانچه مرجع فارسی هم دارید باید یا از بسته Persian-bib استفاده کنید و یا راهنمای bidi را مطالعه کنید.
\bibitem{}
D. Golovaty, L. Berlyand, On uniqueness of vector-valued minimizers of the
Ginzburg-Landau functional in annular domains, {\it Calc. Var.,} {\bf 14}, 213-232, (2002).


\bibitem{1}
Luís Almeida, Threshold transition energies for Ginzburg-Landau 
functionals, {\it Nonlinearity}, Vol.~{\bf 12}, No.~{\bf 5}, 1999, pp.~1389–1414.

\bibitem{2} 
Luís Almeida, Topological sectors for Ginzburg-Landau energies, {\it Rev. Mat. Iberoamericana}., 
Vol.~{\bf 15}, No.~{\bf 3}, 1999, pp.~487–545.


\bibitem{3} 
F. M. Baelus, B. J. Peeters, V. A. Schweigert, Vortex states in superconducting rings,
{\it Phys. Rev. B} Vol.~{\bf 61} , No.~{\bf 14} 2000, pp.~ 9734–9747.


\bibitem{4} 
G. Baym, The microscopic description of superfluidity,{\it  Mathematical Methods in Solid
State and Superfluid Theory (R. C. Clark, G. H. Derrick, eds.) }, Scottish Universities’
Summer School, Oliver and Boyd, Edinburgh, 1967, pp.~ 121–156.


\bibitem{5} 
L. Berlyand, E. Khruslov, Homogenization of harmonic maps and superconducting
composites,{\it  SIAM J. Appl. Math}. Vol.~{\bf 59}, No.~{\bf 5} 1999, pp.~ 1892–1916.


\bibitem{6}
L. Berlyand, E. Khruslov, {\it Homogenization of harmonic maps with large number of
vortices and applications in superconductivity ans superfluidity}, Tech. Report AM212,
CCMA, The Pennsylvania State University, 1999. To appear in Advances in Differential
and Integral Equations.


\bibitem{7}
L. V. Berlyand, K. Voss, {\it Symmetry breaking in annular domains for a Ginzburg-Landau
superconductivity model}, Proceedings of IUTAM99/4 Symposium (Sydney, Australia),
Kluwer Academic Publishers, 1999.

\bibitem{8} 
Fabrice Bethuel, Ha\"ım Br\'ezis, Fr\'ed\'eric H\'elein, Asymptotics for the minimization of
a Ginzburg-Landau functional, {\it Calc. Var. Partial Differential Equations} Vol.~{\bf 1}, No.~{\bf 2} 1993,
pp.~123–148.


\bibitem{9}
Fabrice Bethuel, Ha\"ım Br\'ezis, Fr\'ed\'eric H\'elein, {\it Ginzburg-Landau vortices}, Birkh\"auser
Boston Inc., Boston, MA, 1994.


\bibitem{10} 
Anne Boutet de Monvel-Berthier, Vladimir Georgescu, Radu Purice, A boundary value
problem related to the Ginzburg-Landau model, {\it Comm. Math. Phys}. Vol.~ {\bf 142}, No.~{\bf 1},1991,
pp.~1–23.

\bibitem{11}
H. Br\'ezis, L. Nirenberg, Degree theory and BMO. I. Compact manifolds without boundaries,
{\it Selecta Math. (N.S.)}. Vol.~ {\bf 1}, No.~{\bf 2}, 1995, pp.~197–263.


\bibitem{12}
Ha\"ım Br\'ezis,Yanyan Li, Petru Mironescu, Louis Nirenberg, Degree and Sobolev spaces,
{\it Topol. Methods Nonlinear Anal}. Vol.~ {\bf 13},No.~{\bf 2}, 1999, pp.~181–190.

\bibitem{13}
Ha\"ım Br\'ezis, Luc Oswald, Remarks on sublinear elliptic equations, {\it Nonlinear Anal}.
Vol.~ {\bf 10} No.~{\bf 1},  1986, pp.~55–64.

\bibitem{14}
Myriam Comte, Petru Mironescu, Abifurcation analysis for the Ginzburg-Landau equation,
{\it Arch. Rational Mech. Anal}. Vol.~ {\bf 144}, No.~{\bf 4} ,1998, pp.~301–311.

\bibitem{15}
B. Deaver Jr., W. M. Fairbank, Experimental evidence for quantized flux in superconducting
cylinders, {\it Phys. Rev. Lett}. Vol.~ {\bf 7}, No.~{\bf 2}, 1961,pp.~43–46.

\bibitem{16}
R. J. Donelly, A. L. Fetter, Stability of superfluid flow in an annulus,{\it Phys. Rev. Lett}.
Vol.~ {\bf 17} No.~{\bf 14}, 1966, pp.~747–750.

\bibitem{17}
R. J. Donnelly, {\it Quantized vortices in helium II}, Cambridge Studies in Low Temperature
Physics, Cambridge University Press, Cambridge, 1991.

\bibitem{EV} Lawrence C. Evans, Partial Differential Equations, Second Edition, Graduate Studies in Mathematics, 
AMS, 2010.

\bibitem{bru} Andrew M.Brouckner, Judith B.Brouckner, Brayan S.Thomson, Real Analysis, second Edition, 2008.

\bibitem{ha1} Brezis, Haïm. Degree theory: old and new. {\it Topological nonlinear analysis, II (Frascati, 1995) }, 87--108, Progr. Nonlinear Differential Equations Appl., 27, Birkhäuser Boston, Boston, MA, 1997.

\bibitem{da1} Bernard Dacorogna, Direct Methods in the Calculus of Variations, Second Edition, Applied Mathematical Sciences, Springer, 2007.

\bibitem{da2}  Bernard Dacorogna, Introduction to the Calculus of Variations,ICP, 2004.

\bibitem{ha2}  Ha\"ım Br\'ezis, Functional Analysis, Sobolev Spaces and Partial Differential Equations, Universitex, 2010.

\bibitem{mp} Murray H.Protter, Hans F.Weinberger, Maximum Principles in Differential Equations, Springer New York, 1999. 

\bibitem{18}
David Gilbarg, Neil S. Trudinger, {\it Elliptic partial differential equations of second order} ,
Springer-Verlag, Berlin, 1977, Grundlehren der Mathematischen Wissenschaften, Vol.~ {\bf 224}.

\bibitem{19}
Robert Hardt, David Kinderlehrer, Fang Hua Lin, The variety of configurations of static
liquid crystals,{\it Variational methods (Paris, 1988)}, Birkh¨auser Boston, Boston, MA,
1990,  pp.~115–131.

\bibitem{20}
Shuichi Jimbo, Yoshihisa Morita, Ginzburg-Landau equations and stable solutions in a
rotational domain, {\it SIAM J. Math. Anal}. Vol.~ {\bf 27}, No.~{\bf 5}, 1996, pp.~1360–1385.

\bibitem{21}
I. M. Khalatnikov, {\it An introduction to the theory of superfluids}, Benjamin, 1965.

\bibitem{22}
E. M. Lifshitz, L. P. Pitaevskii, {\it Statistical physics}, part II, Pergamon Press, 1980.

\bibitem{23}
Fang-Hua Lin, Une remarque sur l’application $x/|x|$, {\it C. R. Acad. Sc. Paris}. Vol.~ {\bf 305} ,1987,
 pp.~ 529–531.

\bibitem{24}
Fang-Hua Lin, On nematic liquid crystals with variable degree of orientation, {\it Comm.
Pure Appl. Math}. Vol.~ {\bf 44}, No.~{\bf 4}, 1991,pp.~453–468.

\bibitem{25}
Fang-Hua Lin, Static and moving vortices in Ginzburg-Landau theories, {\it Nonlinear partial
differential equations in geometry and physics (Knoxville, TN, 1995)}, Birkh¨auser,
Basel, 1997, pp.~71–111.

\bibitem{26}
Petru Mironescu, On the stability of radial solutions of the Ginzburg-Landau equation,
{\it J. Funct. Anal}. Vol.~ {\bf 130}, No.~{\bf 2},  1995, pp.~334–344.

\bibitem{27}
Yuri N. Ovchinnikov, Israel M. Sigal,  Ginzburg-Landau equation. I. Static vortices,
{\it Partial differential equations and their applications (Toronto, ON, 1995)}, Amer. Math.
Soc., Providence, RI, 1997, pp.~199–220.

\bibitem{28}
L. M. Pismen, {\it Vortices in nonlinear fields}, Oxford Science Publications, Clarendon
Press, Oxford, 1999.

\bibitem{29}
S. J. Putterman, {\it Superfluid hydrodynamics}, North-Holland Series in Low Temperature
Physics, Vol.~ {\bf 3}, North-Holland Publishing Company, Amsterdam, 1974.

\bibitem{30}
Jacob Rubinstein, Six lectures on superconductivity, Boundaries, interfaces, and transitions
(Banff, AB, 1995), {\it Amer. Math. Soc., Providence, RI}, 1998, pp.~163–184.

\bibitem{31}
Jacob Rubinstein, Peter Sternberg, Homotopy classification of minimizers of the
Ginzburg-Landau energy and the existence of permanent currents, {\it Comm. Math. Phys}.
Vol.~ {\bf 179}, No.~{\bf 1},1996, pp.~257–263.

\bibitem{32}
Etienne Sandier, Sylvia Serfaty, Global minimizers for the Ginzburg-Landau functional
below the first critical magnetic field, {\it Ann. Inst. H. Poincar\'e Anal. Non Lin\'eaire}. Vol.~ {\bf 17},No.~{\bf 1},
2000, pp.~119–145.

\bibitem{33}
Sylvia Serfaty, Local minimizers for the Ginzburg-Landau energy near critical magnetic
field. II, {\it Commun. Contemp. Math}.  Vol.~ {\bf 1}, No.~{\bf 3}, 1999, pp.~295–333.

\bibitem{34}
Sylvia Serfaty, Local minimizers for the Ginzburg-Landau energy near critical magnetic
field. I, {\it Commun. Contemp. Math}. Vol.~ {\bf 1}, No.~{\bf 2} ,1999, pp.~213–254.

\bibitem{35}
Sylvia Serfaty, {\it On a model of rotating superfluids}, Tech. Report 2000-09, CMLA, Ecole
Normale Sup´erieure de Cachan, 2000.

\bibitem{Wo} C.M.~Wood, Some energy-related functionals, and their 
vertical variational theory, Ph.D. Thesis, University of
Warwick, 1983.

% ارجاعات فارسی
 \RTL
\setpersianfont
\footnotesize 
%شروع ارجاعات فارسی

\bibitem{David}
دیوید کیون چنگ، مترجم محمد قوامی، الکترومغناطیس میدان و موج، انتشارات دانشگاه تهران، ایران، ایران، ایران، ایران (۱۳۷۱).

\bibitem{Adams}
حساب دیفرانسیل و انتگرال، مترجم محمد قوامی، الکترومغناطیس میدان و موج، انتشارات دانشگاه تهران، ایران، ایران، ایران، ایران (۱۳۷۲).

\end{thebibliography}







